\documentclass[]{article}

%Change helvetica
\usepackage[scaled]{helvet}
\renewcommand*\familydefault{\sfdefault} %% Only if the base font of the document is to be sans serif
\usepackage[T1]{fontenc}

%Set page width 
\usepackage[textwidth=16cm]{geometry}

%Used to improve the headers
\usepackage{fancyhdr}

%Used to include images into the document
\usepackage{graphicx}

%Used to make some pages landscape
\usepackage{lscape}

%Used to wrap figures in text
\usepackage{wrapfig}

%Translate LaTeX terms to dutch
\usepackage[dutch]{babel}

%Used to display url's
\usepackage{hyperref}

%Used to calculate last page
\usepackage{longtable}

%Used to display "normal" paragraph's without indentation and with blank line
\usepackage[parfill]{parskip}

%Don't break paragraphs in page
\widowpenalties 1 10000
\raggedbottom

%Stop wordbreaking
%\usepackage[none]{hyphenat}

%Use \insertdate to insert current formatted date
\makeatletter
\let\insertdate\@date
\makeatother

%Stop word breaks
\tolerance=1
\emergencystretch=\maxdimen
\hyphenpenalty=10000
\hbadness=10000

%add version command, use \version to insert current version
\newcommand{\version}{2.0}

% Alter some LaTeX defaults for better treatment of figures:
% See p.105 of "TeX Unbound" for suggested values.
% See pp. 199-200 of Lamport's "LaTeX" book for details.
%   General parameters, for ALL pages:
\renewcommand{\topfraction}{0.9}    % max fraction of floats at top
\renewcommand{\bottomfraction}{0.8} % max fraction of floats at bottom
%   Parameters for TEXT pages (not float pages):
\setcounter{topnumber}{2}
\setcounter{bottomnumber}{2}
\setcounter{totalnumber}{4}     % 2 may work better
\setcounter{dbltopnumber}{2}    % for 2-column pages
\renewcommand{\dbltopfraction}{0.9} % fit big float above 2-col. text
\renewcommand{\textfraction}{0.07}  % allow minimal text w. figs
%   Parameters for FLOAT pages (not text pages):
\renewcommand{\floatpagefraction}{0.7}  % require fuller float pages
% N.B.: floatpagefraction MUST be less than topfraction !!
\renewcommand{\dblfloatpagefraction}{0.7}   % require fuller float pages

\pagestyle{fancy}

\fancyhead{}
\fancyfoot{}

\fancyhead[L]{Melroy van den Berg \& Vincent Kriek}
\fancyhead[R]{\includegraphics[height=30pt,keepaspectratio]{tassbw.pdf}}

\fancyfoot[L]{DomoTop - Afstudeerverslag}
\fancyfoot[C]{\thepage\space van \pageref{LastPage}}
\fancyfoot[R]{v\version}

% Add line above footer
\renewcommand{\footrulewidth}{0.4pt}% default is 0pt

\begin{document}

\section{Verantwoording individuele bijdrage}

\subsection{Afstudeerverslag}
\begin{tabular}{|| l | c | c ||}\hline
    Onderdeel              &   \multicolumn{2}{|c||}{Verantwoordelijke} \\\hline
                           & Melroy van den Berg & Vincent Kriek        \\\hline\hline
    Voorwoord              &                     &  X                   \\\hline
    Samenvatting           &       X             &                      \\\hline
    Inleiding              &       X             &                      \\\hline
    De plaats van de...    &                     &  X                   \\\hline
    Probleemanalyse        &                     &  X                   \\\hline
    Plan van Aanpak        &       X             &                      \\\hline
    Methoden en Technieken &       X             &  X                   \\\hline
    Uitvoering             &       X             &                      \\\hline
    Resultaten             &                     &  X                   \\\hline
    Conclusies             &       X             &                      \\\hline
    Aanbeveling            &                     &  X                   \\\hline
\end{tabular}              

\subsection{Proces}
\subsubsection{Keuze van taakverdeling}
\label{taakverdeling}
\textit{Beschrijf waarop de uiteindelijke taakverdeling is gebaseerd?}

De taakverdeling binnen het project is geregeld via de Scrum
ontwikkelmethodiek. In elke sprint werd een lijst met taken opgesteld, waarna
beiden aan een taak begonnen wordt. Als er \'e\'en van de twee klaar is met een
taak, kiest deze een nieuwe en gaat weer verder. 

Voor documentatie hebben we het zoveel mogelijk per hoofdstuk verdeelt of, als
het document te klein was, per document.

\subsubsection{Verloop samenwerking}
\textit{Beschrijf hoe jullie overlegd hebben bij beslissingen en hoe jullie een
discussie of meningsverschil hebben aangepakt.}

Bij meneningsverschillen in het project werden de voor- en nadelen tegen elkaar
afgewogen. Naar aanleiding van deze afwegingen werd er gezamelijk een beslissing
genomen waar verder mee gewerkt kon worden.

Als er problemen of moeilijkheden wordt er hulp gezocht bij de ander. Zo is er
een frisse blik en wordt er samen naar een oplossing toe gewerkt.


\subsubsection{Verdeling o.b.v. van werkzaamheden}
\textit{Beschrijf hoe de verdeling is verlopen op basis van werkzaamheden, zoals
rapportage, interne en externe communicatie, analyse etc.}

De meeste taken werden verdeeld. Er werd gekeken welk van ons het minst te doen
had op het moment en die werd op een bepaalde taak gezet. Zoals verteld in
hoofdstuk \ref{taakverdeling} is er in het project met Scrum gewerkt.


\subsubsection{Op welke punten hebben jullie elkaar versterkt?}
\textit{Was \'e\'en student sterker op een bepaald onderdeel dan de andere?
Licht dit toe.}

Melroy is iemand die veel op details let en dit heeft Vincent veel geholpen.
Zeker met betrekking tot het afstudeerverslag, deze ziet er door dit oog voor de
detail een stuk beter uit.

Vincent is iemand met veel kennis op technisch gebied, waardoor ik (Melroy) met vragen
goed bij hem terecht kan. Op deze manier verloopt de samenwerking ook altijd
goed en worden technische problemen snel opgelost. Ook durft hij
'out-of-the-box' te denken, waardoor we op nieuwe ide\"een komen wat meestal
leidt tot verbeteringen en vernieuwingen hebben binnen ons project. Hij blijft
tevens kalm in stressvolle situaties, wat belangrijk is om goed en helder te
kunnen blijven redeneren en denken. 

\subsubsection{Welke punten waren voor jullie beiden aandachtspunten?}
\textit{Licht dit toe.}

Het beargumenteren van de keuzes die we maakte richting derde partijen had beter
gekund. Er werd soms vergeten de keuzes die gemaakt werden goed te onderbouwen
en uit te leggen. Dit gebeurde zowel in verslagen als in presentaties.

\end{document}

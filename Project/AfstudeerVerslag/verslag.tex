\documentclass[]{article}

%Change helvetica
\usepackage[scaled]{helvet}
\renewcommand*\familydefault{\sfdefault} %% Only if the base font of the document is to be sans serif
\usepackage[T1]{fontenc}

%Set page width 
\usepackage[textwidth=16cm]{geometry}

%Used to improve the headers
\usepackage{fancyhdr}

%Used to include images into the document
\usepackage{graphicx}

%Used to make some pages landscape
\usepackage{lscape}

%Used to wrap figures in text
\usepackage{wrapfig}

%Translate LaTeX terms to dutch
\usepackage[dutch]{babel}

%Used to display url's
\usepackage{hyperref}

\usepackage{longtable}

%Used to display "normal" paragraph's without indentation and with blank line
\usepackage[parfill]{parskip}

%Don't break paragraphs in page
\widowpenalties 1 10000
\raggedbottom

%Stop wordbreaking
%\usepackage[none]{hyphenat}

%Use \insertdate to insert current formatted date
\makeatletter
\let\insertdate\@date
\makeatother

%Stop word breaks
\tolerance=1
\emergencystretch=\maxdimen
\hyphenpenalty=10000
\hbadness=10000

%add version command, use \version to insert current version
\newcommand{\version}{1.13}

% Alter some LaTeX defaults for better treatment of figures:
% See p.105 of "TeX Unbound" for suggested values.
% See pp. 199-200 of Lamport's "LaTeX" book for details.
%   General parameters, for ALL pages:
\renewcommand{\topfraction}{0.9}    % max fraction of floats at top
\renewcommand{\bottomfraction}{0.8} % max fraction of floats at bottom
%   Parameters for TEXT pages (not float pages):
\setcounter{topnumber}{2}
\setcounter{bottomnumber}{2}
\setcounter{totalnumber}{4}     % 2 may work better
\setcounter{dbltopnumber}{2}    % for 2-column pages
\renewcommand{\dbltopfraction}{0.9} % fit big float above 2-col. text
\renewcommand{\textfraction}{0.07}  % allow minimal text w. figs
%   Parameters for FLOAT pages (not text pages):
\renewcommand{\floatpagefraction}{0.7}  % require fuller float pages
% N.B.: floatpagefraction MUST be less than topfraction !!
\renewcommand{\dblfloatpagefraction}{0.7}   % require fuller float pages

\pagestyle{fancy}

\fancyhead{}
\fancyfoot{}

\fancyhead[L]{Melroy van den Berg \& Vincent Kriek}
\fancyhead[R]{\includegraphics[height=30pt,keepaspectratio]{tassbw.pdf}}

\fancyfoot[L]{Afstudeerverslag}
\fancyfoot[C]{\thepage}
\fancyfoot[R]{v\version}

% Add line above footer
\renewcommand{\footrulewidth}{0.4pt}% default is 0pt

\begin{document}

\thispagestyle{plain}

\title{Afstudeerverslag}
\author{Vincent Kriek \and Melroy van den Berg}

\maketitle

\begin{figure}[htpb]
   \begin{center}
     \includegraphics[width=\textwidth]{voorkant.pdf}
   \end{center}
\end{figure}

%\vspace*{-3cm}
\vspace*{\fill}

\begin{tabular}{|| l | l || l | l ||}\hline
   Document ID: & AFVSL-MV-21          &Project naam:     &DomoTop             \\\hline
   Datum:       &14 mei 2012          &School begeleider:&Peter Kailuhu       \\\hline
   Versie:      &\version              &People Manager:   &Gerben Blom         \\\hline
   Status:      &concept               &Documentnaam:     &Afstudeerverslag.pdf\\\hline
   Auteur:      &Melroy van den Berg \&&Reviewer:         &Gerben Blom         \\
                &Vincent Kriek         &                  &                    \\\hline
   Printdatum:  &\insertdate           &Classificatie:    &Openbaar            \\\hline
\end{tabular}

\newpage

\noindent\copyright  TASS B.V. 2011
Alle rechten voorbehouden. Verveelvuldiging, geheel of gedeeltelijk, is
niet toegestaan dan met schriftelijke toestemming van de
auteursrechthebbende.
All rights are reserved. Reproduction in whole or in part is prohibited
without the written consent of the copyright owner.Dit document is
gepubliceerd door:\\
TASS BV\\
Eindhoven, Nederland\\

\noindent Commentaar en suggesties kunnen worden gestuurd naar:\\
\indent TASS B.V.\\
\indent\indent Postbus 80060\\
\indent\indent 5600 KA  EINDHOVEN\\
\indent\indent Nederland\\
\indent\indent tel:  +31 40 2503200\\
\indent\indent fax:  +31 40 2503201\\

\vspace*{\fill}

\section*{Geschiedenis}

\begin{tabular}{|| l | l | l | l ||}\hline
    Versie  &Datum       &Auteur              &Beschrijving                    \\\hline\hline
    0.1     &02-03-2012  &Vincent Kriek \&    &Init opzet                      \\
            &            &Melroy van den Berg &                                \\\hline
    0.2     &02-03-2012  &Vincent Kriek       &Hoofdstuk 2, eerste versie      \\\hline
    0.3     &05-03-2012  &Melroy van den Berg &Plan van Aanpak, Verklarende    \\
            &            &                    &woordenlijst + review           \\
            &            &                    &probleemanalyse                 \\\hline
    0.4     &05-03-2012  &Vincent Kriek       &Probleemanalyse, methoden en    \\
            &            &                    &technieken                      \\\hline
    0.5     &04-04-2012  &Melroy van den Berg &Methoden en technieken verder   \\
            &            &                    &gespecificeerd en uitvoering    \\
            &            &                    &verder uitgebreid               \\\hline
    1.01    &04-04-2012  &Melroy van den Berg &Concept verslag aangekondigd    \\\hline
    1.02    &04-04-2012  &Vincent Kriek       &Resultaten eerste versie        \\\hline
    1.03    &04-04-2012  &Vincent Kriek       &Feedback Peter verwerkt         \\\hline
    1.04    &08-04-2012  &Vincent Kriek       &Eerste versie document in \LaTeX\\\hline
    1.05    &09-04-2012  &Vincent Kriek       &Verslag compleet in \LaTeX      \\\hline
    1.06    &10-04-2012  &Melroy van den Berg &Uitvoering geschreven           \\\hline
    1.07    &25-04-2012  &Melroy van den Berg &Uitvoering uitgebreid           \\\hline
    1.08    &26-04-2012  &Vincent Kriek       &Resultaten uitgebreid           \\\hline
    1.09    &04-05-2012  &Melroy van den Berg &Uitvoering + Methoden \&        \\ 
            &            &                    &Technieken uitgebreid           \\\hline
    1.10    &14-05-2012  &Melroy van den Berg &Conclusie + Inleiding +         \\
            &            &                    &Samenvatting geschreven         \\\hline 
    1.11    &14-05-2012  &Vincent Kriek       &Voorwoord                       \\\hline
    1.12    &21-05-2012  &Vincent Kriek       &Voorwoord verbeterd             \\\hline
    1.13    &24-06-2012  &Melroy van den Berg &Document lay-out verbeterd      \\\hline
\end{tabular}

\newpage
\section*{Voorwoord}
\addcontentsline{toc}{section}{Voorwoord}
Domotica systemen nemen een steeds prominentere rol aan in de huidige
samenleving. Denk bijvoorbeeld aan de Living Colors lampen van Philips, het
KlikAanKlikUit systeem wat je voor twee tientjes bij de Blokker haalt en het
nieuwe Toon van Eneco. Maar wat voor soort beveiliging zit er op zulke systemen?
En wat voor beveiling zou het meest geschikt zijn? In onze stage hebben bekeken
hoe dit soort systemen het best beveiligd kunnen worden.

In de presentatie kan niet alles getoond worden wat er is gedaan en bereikt
tijdens de afstudeerstage. Om alles toch goed toe te lichten hebben we dit
verslag opgesteld.  Onze docenten, gecommiteerde en andere ge"intresseerden
kunnen in dit verslag nalezen hoe wij onze stage hebben aangepakt, welke
methodes en technieken we gebruikt hebben en wat voor resultaten we hebben
bereikt.

Voordat we met de stage begonnen was onze kennis van beveiliging nog beperkt,
maar tijdens onze stage hebben we veel geleerd over dit vakgebied. Samen met
Berry Borgers, \'e\'en van onze technische begeleiders, hebben we een aantal keuzes
gemaakt over hoe we het project aan zouden pakken. Het onderzoek wat hieraan
vooraf ging heeft, samen met de stage, ons veel geleerd over beveiliging.

Het beveiligen van OpenRemote was voor TASS tweeledig. Het is bedoeld als
demonstratieproject op beurzen en scholen. Hiermee is aan te tonen wat er
allemaal mogelijk is op domotica gebied en welke rol beveiliging daar in speelt.
Tevens was het een onderzoeksproject, om te kijken hoe een domoticasysteem
het best beveiligd kan worden zonder in te leveren op het gebied van
gebruiksvriendelijkheid.

We willen Gerben Blom bedanken voor alle begeleiding die we op projectmatig vlak
gekregen hebben. Tevens willen we Berry Borger bedanken voor de technische
beleiding, informatie en advies op security gebied. Tevens willen we Bas Burgers
bedanken voor advies en begeleiding op technisch gebied. Ook willen we onze
begeiding vanuit school bedanken, met dan in het bijzonder Peter Kailuhu voor
de coachingsgesprekken en informatie over de eisen vanuit school.

\newpage
\tableofcontents
\newpage
\listoftables
\listoffigures

\newpage
\section*{Samenvatting}
\addcontentsline{toc}{section}{Samenvatting}
Het software project genaamd OpenRemote is uitgebreid met een beter
beveiligingssysteem dan er voorheen inzat. OpenRemote is een software
pakket om een domotica systeem op te zetten. Er wordt gebruik gemaakt van
een PlugTop computer om hierop de OpenRemote Controller te draaien, wat functioneert als
zijnde een server. Vandaar de projectnaam DomoTop (Domotica PlugTop).
De beveiligingtechniek die gebruikt wordt in het DomoTop project is SSL/TLS,
waarin het bijzonder gebruik wordt gemaakt van client certificaten om de
gebruikers/apparaten te authenticeren. Ook is er later in het DomoTop project
voor gekozen om de mogelijkheid te geven om groepen aan te maken waardoor
het het mogelijk is om apparaten te plaatsen in een specifieke groep. Hierdoor
is het dus mogelijk om apparaten te autoriseren.

Deze opdracht is afkomstig en bedacht door het bedrijf TASS. De opdracht wordt
begeleid door de people manager Gerben Blom, technische begeleiders Berry
Borgers en Bas Burgers. Deze stageopdracht is gevonden via de school Avans Den
Bosch bij een bedrijven beurs te Rotterdam. De opdracht is aangenomen door
Melroy van den Berg \& Vincent Kriek.

\begin{figure}[htpb]
   \begin{center}
     \includegraphics[width=0.7\textwidth]{TopLevelDesignORSecurity.pdf}
   \end{center}
   \caption{Flowchart initial request}
\end{figure}

TASS vond het belangrijk om OpenRemote software te beveiligen en een oplossing
te vinden van een authenticatie mechanisme zonder gebruik te hoeven maken van
een gebruikersnaam/wachtwoord. Tevens was het belangrijk dat er ervaring en
onderzoek werd gedaan naar een PlugTop computer, waarop de OpenRemote Controller
gaat draaien. Tot slot was ook onderzoek nodig naar verschillende Domotica
protocollen en verschillende manieren van beveiligen.  Hierdoor is er kennis en
ervaring opgedaan op gebied van onder andere domotica, Google Android, Embedded
Linux, maar ook zijn er verschillende onderzoeksrapporten geschreven met
onderzoekresultaten. De meeste kennis is verkregen op gebied van SSL/TLS,
waarbij onderzoek naar gedaan is om het uiteindelijk te kunnen implementeren. 

Op dit moment is OpenRemote beveiligd via client certificaten en kan de
administrator via de OpenRemote administrator panel apparaten goed-/afkeuren. De
administrator panel is een webbased applicatie en behoord tot de OpenRemote
Controller. Op de OpenRemote Modeler kunnen er groepen gemaakt worden en
gekoppeld worden aan knoppen, sliders en/of switches, waarna je deze groepen in
de administrator panel kan koppelen aan apparaten.

\newpage
\section{Inleiding}
Vandaag de dag merk je dat beveiliging een steeds belangrijkere rol is binnen
de samenleving, er wordt immers veel gesproken over beveiliging kwesties in
het nieuws. Zo is KPN onlangs wederom weer gehackt en dit keer door een 17
jarige jongen of
er worden weer allerei persoonlijke gegevens en/of logingegevens verspreid via het
internet.
In dit verslag wordt besproken hoe de beveiliging is gesteld met een
softwareapplicatie genaamd OpenRemote. OpenRemote is een applicatie die
bedoeld om thuis een (eigen) domotica systeem op te zetten. Hoewel er wel een
Hoewel de mogelijkheid bestaat om een vorm van beveiling aan te zetten is deze
erg beperkt. Daarom wordt er gekeken naar betere
alternatieven om dit product te beveiligen en later uit te breiden. Het zou een
nare ervaring zijn als ineens je buren bijvoorbeeld je lampen in jouw huis aan en uit
kunnen doen.

In dit verslag worden bovenstaande punten besproken verdeeld in verschillende
hoofdstukken, zoals onder andere probleemanalyse, methoden en technieken en resultaten.
Tot slot eindigd dit verslag het met een conclusie en aanbevelingen.

\newpage
\section{De plaats van de afstudeerder binnen de organisatie}

TASS is 30 jaar geleden begonnen als onderdeel van Philips. Het is tot 2007
onderdeel gebleven van Philips waarna het zelfstandig is verder gegaan,
onder de vleugels van moederbedrijf Total Specific Solutions (TSS).

\begin{wrapfigure}{r}{0.5\textwidth}
  \begin{center}
    \includegraphics[width=0.40\textwidth]{tass_eindhoven.pdf}
  \end{center}
  \caption{TASS Eindhoven}
\end{wrapfigure}

TASS is een detacheerder wat inhoudt dat hun personeel werkzaam is bij
andere bedrijven. Zeker gezien hun geschiedenis heeft TASS nog een goede
relatie met vorig moederbedrijf Philips en werkt dus ook veel met Philips
samen. Andere bedrijven waar veel mee wordt samengewerkt zijn TomTom, ASML
en Bosch.

Buiten het detacheren heeft TASS ook een aantal projecten intern lopen. Het
grootste en meest bekende project is het uCAN[1] project. Dit project
focust zich op het inzichtelijk maken van sensorgegevens uit auto's. Het
leest de CAN-Bus uit en zendt de informatie door naar een server in de
cloud.

Binnen TASS is het grootste deel van het personeel gedetacheerd bij andere
bedrijven. Deze medewerkers krijgen allemaal een people manager toegewezen.
Deze people manager zorgt voor de relatie tussen TASS en de medewerker evenals
eventuele problemen tussen de medewerker en het bedrijf waar deze
gedetacheerd zit.

De plek waar de medewerker gedetacheerd is geregeld door de account
managers van TASS. Een account manager heeft een aantal bedrijven in
zijn/haar portfolio zitten met wie ze contact hebben over mogelijke
opdrachten voor medewerkers. Mochten ze een plek vinden, zorgen zij voor
een goede kandidaat in samenwerking met de people managers.

Tijdens het afstuderen is er voor de studenten ook een people manager
die het project zal begeleiden, in dit geval is het Gerben Blom. Deze
people manager zal de afstudeerders begeleiden op projectmatig vlak,
functioneren als Project Owner en eventueel antwoord geven op technische
vragen.

Twee wekelijks zal er een review punt komen waar de people manager
samenkomt met de afstudeerder om de voortgang en problemen te bespreken.
Tevens zal daar besproken en getoond worden wat voor producten er deze week
opgelevererd zijn. Hierna zullen ook de taken van de volgende twee weken
bepaald worden en wordt er gekeken of daar haken en ogen aan zitten.

Buiten de People Manager krijgen de afstudeerders ook een technische
begeleider toegewezen. Onze hoofd technische begeleider heet Bas Burgers,
maar ook Berry Borgers is erg ge\"interesseerd is het DomoTop project en hij
wilt graag de tweede technische begeleider zijn. De people manager is
niet meer dagelijks bezig met technische ontwikkeling en kan op dat vlak
eventueel achter lopen. De technische begeleider zal een
gedetacheerde medewerker zijn die nog wel dagelijks bezig is met techniek.
Deze technische begeleider zal zich dan ook niet bezig houden met de
project management zaken maar vooral met de technische problemen of vragen
waar tegenaan gelopen wordt.

\begin{wrapfigure}{l}{0.5\textwidth}
  \begin{center}
    \includegraphics[width=0.20\textwidth]{openremote.pdf}
  \end{center}
  \caption{OpenRemote logo}
\end{wrapfigure}

De klant zal in dit project vertegenwoordigd worden door de People Manager
Gerben Blom. Dit zal in dit project niet de enige klant zijn, ook de
ontwikkelaars en gebruikers van het OpenRemote project de rol van klant
fungeren. Dit betekent dat de documentatie van het project hierop aangepast
moet worden, er zal moeten worden gedocumenteerd in het Engels.  
Wij zouden graag zien dat de uitbreiding opgenomen zullen worden in OpenRemote
en daarvoor moet het project aan de standaarden voldoen die OpenRemote stelt. We zullen
verantwoording over de keuzes moeten afleggen aan de ontwikkelaars van
OpenRemote. Tevens kunnen de ontwikkelaars van OpenRemote ons helpen en begeleiden
in het proces en dienen als begeleider in het project.

Tot slot zijn de afdelingen HR en ICT beheer betrokken met het project. Bij
de HR afdeling worden zaken zoals contracten behandeld. Bij de ICT beheer
afdeling binnen de organisatie moet er gedacht worden aan het bestellen van
hardware, computers opzetten en bij computer gerelateerde problemen of
vragen kan men bij deze afdeling terecht.

\newpage
\section{Probleemanalyse}
Binnen de opdracht waren er een probleemsituatie onstaan die hieronder
beschreven worden. Naar aanleiding van deze probleemsituaties is er een
probleemstelling opgesteld.

\subsection{Probleemsituatie}

Het open source domotica platform OpenRemote is een software pakket wat
gebruikers in staat stelt een domotica systeem op te zetten in huis wat
overweg kan met meerdere domotica protocollen. Dit systeem bestaat uit drie
delen, ten eerste de OpenRemote controller. Deze controller is een stuk
software wat draait op een computer binnen het huis wat door middel van
domotica geautomatiseerd gaat worden.

De controller communiceert rechtstreeks met het tweede onderdeel van het
OpenRemote pakket, namelijk de client applicatie. Er is een Android, iOS en
webapplicatie te downloaden die kan werken met de OpenRemote controller.

De gebruikersinterface van deze applicaties wordt, samen met de logica
erachter, ontworpen via het derde onderdeel: de composer. Deze composer is
een web applicatie die de mogelijkheid geeft tot het ontwerpen van de
client applicatie evenals de het configureren van de actie of acties die
gebeuren als er bijvoorbeeld een knop wordt ingedrukt.

Dit software pakket heeft echter een groot gebrek, het is niet beveiligd
afgezien van een globale gebruikersnaam en wachtwoord voor het hele
systeem. Dit is niet gewenst en dit opdracht is om dit te verhelpen door
er een authenticatie en autorisatie mechanisme aan toe te voegen.

\subsection{Probleemstelling}

Het probleem wat opgelost moet worden is dat onbevoegden makkelijk toegang
kunnen krijgen op een OpenRemote systeem. Voor dit probleem is de
volgende doelstelling geformuleerd: Het implementeren van beveiliging in
OpenRemote, zodat onbevoegden geen toegang kunnen krijgen op het systeem.

Het eerste deelprobleem die te defini"eren is over deze probleemstelling is:
Welke beveiligingstechnieken zijn het best te gebruiken in deze situatie?
Deze doelstelling zal onderzocht moeten worden en uiteindelijk moet het
onderzoek ge\"implementeerd worden in OpenRemote.

Het tweede deelprobleem wat opgelost moet worden is: De ontwikkelaars
moeten de code van OpenRemote doorgronden om deze uiteindelijk uit te
kunnen breiden. Voordat er aan de slag kan worden gaan met uitbreiding van
OpenRemote moet de code eerst onderzocht worden.

De volgende probleemstelling die opgelost moet worden binnen het project
is: Geselecteerde gebruikers moet toestemming tot het systeem kunnen worden
verleend. Deze probleemstelling gaat meer richting de autorisatie en over
hoe men gebruikers toestemming kan geven tot een systeem.

Het laatste probleem wat gedefinieerd kan worden is: Er moeten groepen gemaakt
kunnen worden om rechten aan gebruikers te geven. Op deze manier kunnen knoppen
alleen toegankelijk zijn voor een deel van de gebruikers. 

Voor dit project zal SCRUM gebruikt worden als ontwikkelmethodiek, omdat het
daarmee makkelijk is een lopend project door te ontwikkelen. Voor meer
informatie zie het hoofdstuk “Methoden en Technieken”.

\newpage
\section{Plan van Aanpak}

In dit hoofdstuk wordt er een samenvatting gegeven van het plan  van  aanpak
, zie het volledige plan van aanpak in bijlage 1.  Het  doel  van  het
project is  van  het  om  OpenRemote  te  beveiligen  en  dit  eventueel  te
demonstreren.  De  opdracht  binnen  het  project  is   om   OpenRemote   te
beveiligen. Op dit moment is er geen enkele manier van beveiliging  aanwezig
in OpenRemote. Ook zal er gebruik gemaakt worden van een PlugTop, waarop  de
OpenRemote server komt te draaien. Er zal gebruik gemaakt worden van  SCRUM,
een ontwikkelmethode, wat flexibel is en resulteert in een product dat  veel
meer aan de verwachtingen van de klant voldoet. Meer informatie  over  SCRUM
kan gevonden worden in het hoofdstuk 'Methoden en technieken'.
Er wordt git gebruikt als versiebeheer software en  dat  in  combinatie  met
GitHub, waar de git hosting online geregeld is.  Er  zal  Google  Documenten
gebruikt worden, waarin de documenten geschreven, opgeslagen worden.
Al deze technieken komen ook verder aan bod in het  hoofdstuk  'Methoden  en
technieken'.

In de eerste weken wordt  er  onderzoek  gedaan  over  welke  technieken  er
gebruikt gaan worden, welk platform,  PlugTop  en  protocollen  er  gebruikt
gaan worden. Daarna wordt de PlugTop opgezet samen met een  mobiel  apparaat
(Android)  waarop  OpenRemote  draait.  Er  wordt  een  'Research  Security'
document opgezet, hierin komt een conclusie te staan wat de aanpak wordt  om
OpenRemote het beste  te  beveiligen.  Met  deze  informatie  wordt  er  een
prototype ontwikkeld. Nadat er een volledig werkend  prototype  is,  kan  er
gekeken worden of en waar dit product uitgebreid en verbeterd kan worden.

Zie bijlage 1 voor het volledige plan van aanpak met daarin een lijst  van
risico's,  afspraken,  deadlines  en  een weekplanning.

\newpage
\section{Methoden en Technieken}

Binnen het project zijn er een aantal methoden, technieken en tools
gebruikt worden om tot een goed eind te komen. Hier zullen de
verschillende keuzes voor methoden, technieken en tools toegelicht worden.

\subsection{Ontwikkelmethodiek}
De eerste keuze was voor een ontwikkelmethodiek. De ontwikkelmethodiek
moest een duidelijke voortgang kunnen laten zien door het project heen. De
begeleider vanuit school moet goed op de hoogte gehouden kunnen worden van
de voortgang evenals de bedrijfsbegeleider. SCRUM is, met de review
momenten, hier een mooie techniek voor. Tevens is binnen het bedrijf veel
kennis en expertise over SCRUM aanwezig.

\begin{figure}[htpb]
  \begin{center}
    \includegraphics[width=0.80\textwidth]{scrum.pdf}
  \end{center}
  \caption{scrum}
\end{figure}

TASS heeft een interne website[2] opgezet, genaamd Sensei, waarin
een grote bron aan informatie over SCRUM is opgenomen. Tevens zijn er een
aantal personen binnen het bedrijf SCRUM-experts wat betekent dat vragen of
problemen altijd beantwoord kunnen worden door een van de experts.

Voor de verschillende SCRUM backlogs is ervoor gekozen een spreadsheet te
gebruiken. Er is gekeken naar verschillende software oplossing maar geen
voldeet aan de verwachtiging. Een eigen spreadsheet heeft veel
flexibiliteit, alles wat er nodig zou zijn kan zelf ingebouwd worden.

Tevens zijn er meerdere burndown charts om de voortgang snel in te kunnen
zien en bij te kunnen houden. Deze burndown charts houden verschillende
elementen van de backlogs bij om op deze manier alles inzichtelijk te
houden.

\newpage
\subsection{Versiebeheer}

\begin{wrapfigure}{l}{0.3\textwidth}
  \begin{center}
    \includegraphics[width=0.20\textwidth]{git.pdf}
  \end{center}
  \caption{Git}
\end{wrapfigure}
De techniek gebruikt wordt om code en document versies bij te houden is
git. Git is een “version control system” net als bijvoorbeeld subversion en
mercurial. Het doel van git is om verschillende versies van bestanden bij
te houden en te kunnen springen in versies. Git heeft als grote voordeel
dat het gedistribueerd is, wat betekent dat elke git clone (vergelijkbaar
met een svn checkout) een volledige geschiedenis met zich mee heeft. Het is
dus niet noodzakelijk een centrale server te gebruiken, al wordt dit vaak
wel gedaan. Dat betekent ook dat elke git clone een backup is van de gehele
geschiedenis.

Om toch een centrale plaats te hebben waar alle verandering naartoe
gestuurd te worden hebben wij ervoor gekozen om Github te gebruiken. Github
is een online platform voor git-repositories, met een erg goede
webinterface. In deze webinterface kan de code ingezien worden, issues
worden aangemaakt en is er een wiki aanwezig.

\subsection{Documenten}

\begin{wrapfigure}{r}{0.5\textwidth}
  \begin{center}
    \includegraphics[width=0.20\textwidth]{docs.pdf}
  \end{center}
  \caption{Google Documenten}
\end{wrapfigure}
Voor het maken van documenten is er voor gekozen om Google Documenten te
gebruiken. Google Documenten is een web applicatie waarmee eenvoudig en
snel documenten kunnen worden gemaakt. Tevens is het mogelijk om met
meerdere mensen in een document te werken zonder conflicten te krijgen. Ook
de SCRUM spreadsheets staan Google Documenten.

Voor het afstudeerverslag is er voor \LaTeX\space gekozen. Dit omdat het
makkelijker is om de opmaak en uiterlijk te wijzigen en dit meer mogelijkheden
biedt dan het meer beperkt Google Documenten. Ook bestond de wens bij studenten
om meer ervaring op te doen met \LaTeX\space en dit leek hen de de uitgelezen kans om
dit te doen

\subsection{Programmeertalen}
De talen die gebruikt gaan worden liggen al vast aangezien er verder 
gewerkt wordt op een bestaand project. Voor de controller wordt gebruik gemaak van
Java, evenals voor de composer. De verschillende applicaties maken gebruik
van Java op Android en Objective-C op iOS.

\newpage
\subsection{Database}
Er is gebruik gemaakt van RazorSQL[3]. Met dit programma is het mogelijk om
SQL queries te maken, typen en de testen via een grafische interface en dit
alles werkt ook met de database die gebruikt wordt bij het DomoTop project,
genaamd HSQLDB[4]. In deze database worden gegevens opgeslagen van de
apparaten die zich aanmelden bij OpenRemote evenals de groepen.
RazorSQL is enkel een tool die gebruikt is bij het
testen/opzetten van de database, daarom is deze tool niet meer
noodzakelijk.


\begin{wrapfigure}{r}{0.4\textwidth}
  \begin{center}
    \includegraphics[width=0.30\textwidth]{hsql.pdf}
  \end{center}
  \caption{HyperSQL}
\end{wrapfigure}

Er is gekozen voor HSQLDB vanwege het feit dat deze database 100\% in Java
is geschreven (net als de Controller van OpenRemote), volledig open-source is,
relatief weinig impact heeft op het system en multithreaded is.
HSQLDB is namelijk klein en gebruikt weinig bronnen. Tot slot voldoet
HSQLDB aan alle eisen en beschikt over alle mogelijkheden die noodzakelijk
zijn binnen het DomoTop project, denk hierbij aan tabellen aanmaken, rijen
toevoegen, updaten, verwijderen en veel meer. Alternatieven van HSQLDB zijn
niet in native Java geschreven,  vraagt meer van het systeem qua
bronnen en zijn groter dan HSQLDB. Sommige alternatieven zijn zelfs niet
eens geschikt om te gebruiken in combinatie met een TomCat server.

\subsection{Ontwikkelomgeving}
Voor de Android applicatie wordt gebruik gemaakt van de Eclipse
ontwikkelomgeving met de ADT plugin. Hiermee is het makkelijk om wijziging
live te draaien op een tablet of een emulator.

\begin{wrapfigure}{r}{0.5\textwidth}
  \begin{center}
    \includegraphics[width=0.20\textwidth]{eclipse.pdf}
  \end{center}
  \caption{Eclipse}
\end{wrapfigure}

Bij het programmeren van de OpenRemote Controller wordt er ook gebruik gemaakt
van de Eclipse omgeving. OpenRemote Controller is in Java geschreven code
en deze Controller is ook vanaf het begin af aan gestart in de Eclipse
ontwikkeling omgeving. Omdat OpenRemote Controller al ontwikkeld is in
Eclipse, wordt er ook gebruik gemaakt van Eclipse. Er wordt gebruik gemaakt van Ant om de Controller te bouwen en compileren. Uiteindelijk krijg
je één WAR bestand, deze WAR kan eenvoudig geïmporteerd worden in de TomCat
Web Application Manager waarna de Controller gebruikt kan worden. TomCat is
de webserver die OpenRemote al gebruikt om de OpenRemote Controller op te
kunnen zetten. Het DomoTop project heeft daarom ook TomCat gebruikt, omdat
dit het meest toegankelijke oplossing is, gezien het feit OpenRemote hier
ook al mee werkt.

Tevens wordt er gebruik gemaakt van een Eclipse plugin genaamd Colorer[5].
Met deze plugin is het mogelijk om syntax highlighting aan te zetten voor
XML/HTML documenten, maar ook CSS en JavaScript bestanden. Er is hiervoor
gekozen, omdat Eclipse standaard geen highlighting heeft voor bovengenoemde
type bestanden. Desalniettemin is het optioneel om de Eclipse Colorer
plugin te gebruiken, het is enkel gekozen voor gemaks doeleinden.

\subsection{Libraries}
\begin{figure}[htpb]
   \begin{center}
     \includegraphics[width=0.6\textwidth]{freemarker.pdf}
   \end{center}
   \caption{FreeMarker}
\end{figure}

In de OpenRemote Controller is er gebruikt gemaakt van een template engine
genaamd Freemarker[6]. Deze template engine maakt het mogelijk om statische
opmaak (template design) te scheiden van dynamisch gegevens zoals apparaat
gegevens uit een database. Freemarker werkt goed samen met Java en TomCat.
Verder is FreeMaker een relatief kleine template engine en simpel te
begrijpen maar toch doeltreffend. FreeMarker forceert zich op het gebruik
van een Model View Controller software architectuur. FreeMarker wordt ook
vaak gebruikt voor servlet gebaseerde web applicatie ontwikkeling en dat is
exact wat de OpenRemote Controller is.

\begin{wrapfigure}{r}{0.5\textwidth}
  \begin{center}
    \includegraphics[width=0.55\textwidth]{bouncyjava.pdf}
  \end{center}
  \caption{Bouncy Castle}
\end{wrapfigure}
Om te kunnen werken met encryptie en TLS te kunnen werken in native Java code
wordt er gebruik gemaakt van Bouncy Castle wat een gratis open-source software.
Bouncy Castle is een bibliotheek die binnen het project onder andere gebruikt
wordt om nieuwe certificaten te kunnen generen vanuit Java code alsmede een CSR
bestand uit te lezen en deze gegevens gebruiken om een nieuw certificaat te
onderteken door middel van een CA. Ook het gebruikte CA certificaat wordt
aangemaakt door middel van Bouncy Castle. Ook onder Android wordt er gebruik
gemaakt van de Bouncy Castle bibliotheek, echter heet dat onder Android Spongy
Castle, maar is exact dezelfde code.

\newpage
\section{Uitvoering}

Het begin van de stage stond in het teken van onderzoek. Er was bij het
ontwikkelteam weinig kennis van het domotica veld, OpenRemote, PlugTop computers
en beveiliging dus al deze velden moesten onderzocht worden. Het project begon
met een globaal onderzoek naar het domotica veld en PlugTop computers. Het
volgende onderzoek wat uitgevoerd was, was naar het OpenRemote software pakket en
eventuele alternatieven.

Het laatste onderzoek is gedaan naar de beveiligingstechnieken die bestaan.
Voordat het onderzoek begon, is er een gesprek gehouden met Berry Borgers
aangezien hij over veel kennis beschikt met betrekking tot beveiling. De keuzes
die zijn gemaakt in het document zijn in overleg met Berry Borgers gebeurd. Dit
onderzoek is gedocumenteerd in het Engels zodat het te delen is met de
ontwikkelaars van OpenRemote. Alle onderzoeken zijn  te vinden in de bijlages.

Vervolgens is er begonnen met het ontwikkelen van de gekozen oplossing:
TLS implementeren in OpenRemote om deze veiliger te maken. Dit gebeurt
door een apparaat een aanvraag te laten doen bij de OpenRemote Controller. De
OpenRemote Controller kan deze aanvraag goedkeuren en het apparaat ontvangt
vervolgens het client certificaat. Dit client certificaat wordt vervolgens
gebruikt in combinatie met een apparaat (zoals Android), waarna er toestemming
wordt verleend om de OpenRemote Controller te gebruiken.

Het was noodzakelijk om zowel de OpenRemote Controller alsmede de OpenRemote
applicatie van Android te modificeren. De aanpak was om allereerst alles op de
server, waar ook de OpenRemote Controller draait, de certificaten te maken en te
genereren. Het CSR bestand wordt aangemaakt en ondertekend op de server en ook
het client certificaat wordt gegenereerd op de server. Om dit mogelijk maken
wordt gebruik gemaakt van het openssl commando onder Linux.  De certificaten die
gegenereerd worden, worden toegevoegd aan de truststore van TomCat. Echter bleek
snel dat TomCat geherstart moest worden om de truststore opnieuw uit te 
kunnen lezen. Dit betekent dat als er een nieuwe client aangemeld wordt, TomCat
geherstart moest worden.
\\ Om dat probleem op te lossen, zodat TomCat server niet
telkens opnieuw herstart hoefde te worden, is gekozen om een eigen
'self-signed' CA te gebruiken. Op deze manier hoeft alleen het certificaat van
de CA toegevoegd te worden aan de truststore van TomCat.  Zolang de client
certificaten ondertekend zijn door een CA certificaat uit de
truststore wordt deze vertrouwd. Als een client vertrouwd wordt, kan deze
gebruik maken van de OpenRemote Controller.

Toen het bovenstaande naar behoren werkte, is er verder gegaan met het
optimaliseren en verfijnen van de OpenRemote Android applicatie en OpenRemote
Controller. In de Android applicatie is er gewerkt aan het automatisch sturen
van CSR bestanden naar de Controller. Tevens zoekt de Android applicatie
automatisch naar het client certificaat zodra er gedrukt wordt op 'Done'. Tot
slot zijn er verschillende bugs opgelost, waaronder enkele stabiliteits
problemen.

%Dit is meer Resultaten, moet technischer
In de OpenRemote Controller is er gewerkt aan verschillende kleine
verbeteringen, zoals client certificaten automatisch laten verwijderen uit zowel
de key store als database bij het drukken op de 'Reset devices' knop. Maar ook
wordt de CA en database automatisch aangemaakt aan het begin dat de applicatie
wordt ge\"installeerd.  Een apparaat kan nu ook volledig verwijderd wordt uit het
systeem, vervolgens is het nu mogelijk om de authenticatie via Client
certificate aan en uit te zetten. 

%Nooit uitgelegd dat het zo werkt
Wanneer er geen rechtstreeks toegang is tot het internet op de 
OpenRemote Controller is het noodzakelijk om toch in te kunnen loggen met de
logingegevens van de Beehive server. Er is daarom gekozen bij de login module
gebruik te maken van de cache gegevens uit de database om de administrator te
kunnen valideren.

Het is van groot belang dat deze gegevens veilig worden opgeslagen. Daarom is er
gekozen voor het opslaan van het wachtwoord in SHA512 (behoord tot de SHA-2
familie). Ook wordt bij de HTTP sessie id  wordt er gebruik gemaakt van het SHA512
hash algoritme. Voorheen was dit MD5 echter is een MD5 hash relatief
eenvoudig te kraken via brute-force en/of rainbow table. Het voordeel van SHA512
is dat het gebruik maakt van 512 bits woord in tegenstelling tot 128
bits, waardoor het veel minder snel gekraakt kan worden.

Er is gebruik gemaak van een "Pepper \& Salt". Met deze termen wordt bedoeld dat
aan het wachtwoord met een bepaalde random lijst met karakters wordt uitgebreid
(Salt) en een geheime constante (Pepper) wordt toegevoegd. Er is gekozen om een
systeem constante te gebruiken als zijnde 'Pepper' en een random lijst van
karakters die gegeneerd wordt via het SHA-512 algoritme met een lengte van 32
bits die gebruikt wordt als 'Salt'. Deze methode zorgt ervoor dat wachtwoorden
lastiger te achterhalen zijn. 

\begin{wrapfigure}{l}{0.5\textwidth}
  \begin{center}
    \includegraphics[width=0.30\textwidth]{tomcat.pdf}
  \end{center}
  \caption{Het TomCat logo}
\end{wrapfigure}

%UITLEG, HOE GEBEURD DIT?
De client certificaten worden niet meer gegenereerd op de server vanwege
veiligheidsredenen. Als de certificaten overgestuurd worden, wordt ook
de private key over de lijn gestuurd wat kan resultering in onderschepping van
deze sleutel. Het is dus verstandig om zoveel mogelijk acties lokaal op android uit te
voeren. Om dit tot een goed einde te brengen is is het noodzakelijk
dat er onder Android onderzocht wordt hoe er een client CSR bestand gemaakt kan
worden en hoe het client certificaat vervolgens ge\"importeerd kan worden in de
keystore van Android. Op de server, waar ook de OpenRemote Controller draait, werd
er gekeken hoe een CA aangemaakt kan worden voor de TomCat server, hoe client
certification request files omgezet kunnen wordfen in normale client certificaten die
ondertekend zijn door het CA certificaat. En hoe de client certificaten
vervolgens opgeslagen kunnen worden in \'e\'en keystore op de server, zodat er op een langer
termijn nog steeds informatie opgevraagd kan worden over een specifiek client certificaat.
Op de server en de OpenRemote controller werkte dit eerst via het OpenSSL
commando afgehandeld, later is deze code omgeschreven in native Java code en
wordt er gebruik gemaakt van de Bouncy Castle library.

%Moet omhoog, is gemaakt voor SHA shizzle
Tenslotte is er een grafische interface gemaakt (te zien in figuur 12) om
apparaten goed en/of af te keuren vanuit de 'administrator panel'. Er bestaat
een pin die vergeleken kan worden met de pin die aanwezig is op het Android
apparaat. Indien de pin van Android en de administrator panel overeenkomen, is
er met zekerheid te zeggen dat het hetzelfde apparaat betreft. Deze pin is
gebaseerd op de publieke sleutel die ook aanwezig is in het client certificaat.
Naast 'User management' heeft de administrator panel ook een tab
'Configuration'. In deze tab is het mogelijk om bepaalde instellingen te
veranderen zoals de authenticatie of pin check aan- of uit te zetten.  De
administrator panel is beveiligd door een gebruikersnaam en wachtwoord die
overeen moet komen met een bestaand OpenRemote Designer account. Op deze manier
is het administratie paneel enkel toegankelijk voor de administrator.

\begin{figure}[htpb]
   \begin{center}
     \includegraphics[width=0.6\textwidth]{userlist.pdf}
   \end{center}
   \label{userlist}
   \caption{Gebruikers management lijst}
\end{figure}

Er is veel vraag vanuit de OpenRemote community om apparaten in groepen onder te
kunnen brengen, hier is gehoor aangegeven. Binnen het project stond al op de
planning om deze mogelijkheid te implementeren. In de OpenRemote Composer (ook
wel Modeler genoemd) is het nu mogelijk om groepen toe te voegen en knoppen,
sliders en switches te koppelen aan een groep. Vervolgens wordt het gegenereerde
XML bestand uitgelezen in de OpenRemote Controller, waarna de lijst met groepen
in de database wordt opgeslagen. Via de administrator panel is het mogelijk om
apparaten te koppelen aan een bepaalde groep zoals te zien in figuur 13. Zodra
de gebruiker van een apparaat op bijvoorbeeld een knop drukt, controleert de
OpenRemote Controller door middel van het client certificaat tot welke groep de
client behoord om vervolgens te controleren of deze groep ook gekoppeld is aan
de knop die ingedrukt was. Tevens is er een optie om groepen niet te
verplichten, dit kan ingesteld worden via de 'Configuration' tab in de
administrator panel. Wanneer groepen niet verplicht zijn, heeft het apparaat
automatisch toegang tot alle mogelijke commando's naar de server. Dit betekent
echter niet dat authenticatie niet vereist is, dit kan nog steeds nodig zijn.

\begin{figure}[htpb]
   \begin{center}
     \includegraphics[width=0.6\textwidth]{usergroups.pdf}
   \end{center}
   \caption{Gebruikers koppelen aan groepen}
\end{figure}

\newpage
\section{Resultaten}

Tijdens de afstudeerperiode zijn er resultaten geboekt die vermeld zijn in dit
hoofdstuk. Hier wordt per iteratie uitgelegd welke resultaten behaald
zijn en aan welke eisen deze moest voldoen.

\subsection{Iteratie 1}
De eerste iteratie is onderzoek gedaan in het werkveld van de afstudeerstage.
Er is onderzoek gaan naar vier delen: Domotica protocollen, domotica hardware,
PlugTops en OpenRemote met alternatieven. 

\begin{table}[htpb]
  \caption{Eisen iteratie 1}
  \begin{center}
    \begin{tabular}{|| l ||}\hline
        Eis                                                  \\\hline\hline
        Uitzoeken protocollen en onderzoeksdocument opzetten \\\hline
        Uitzoeken PlugTop en onderzoeksdocument opzetten     \\\hline
        Uitzoeken Hardware en onderzoeksdocument opzetten    \\\hline
        Uitzoeken OpenRemote en onderzoeksdocument opzetten  \\\hline
    \end{tabular}
  \end{center}
\end{table}

Het doel van deze iteratie is het scheppen van opheldering over de mogelijke
keuzes. Er is onderzocht welke domotica protocollen er bestaan en welke het
meest geschikt zijn voor toepassing in het project. Uit dit onderzoek is gekomen
dat X10 de meest gebruikte, voordeligste en veelzijdigste protocol op de markt
is. Ook is Z-Wave een mooi protocol aangezien het draadloos is. Echter is deze
duurder en wordt deze minder goed ondersteund. Dit betekent dat het voor dit
project minder geschikt is, dus de keuze is op X10 gevallen.  

Het PlugTop onderzoek heeft aangetoond dat de binnen TASS gebruikte
DreamPlug de meest geschikte PlugTop voor dit project is. De DreamPlug
beschikt over een groot aantal aansluiting en is gunstig geprijsd. 

In het onderzoek naar OpenRemote is gekeken naar de mogelijkheden van het
OpenRemote project in vergelijking met andere software pakketten. Hier is uit
gebleken dat OpenRemote beschikt over de meeste mogelijkheden en het meest
actief wordt bijgehouden. Wel is de beveiliging van OpenRemote niet heel
goed geregeld. In bijlage X is het onderzoek te vinden.

\subsection{Iteratie 2}
\begin{table}[htpb]
  \caption{Eisen iteratie 2}
  \begin{center}
    \begin{tabular}{|| l ||}\hline
        Eis                                                              \\\hline\hline
        Als gebruiker wil ik een Android applicatie om een lamp aan en   \\ 
        uit zetten                                                       \\\hline
        Als klant wil ik weten wat voor authorisatie en authenticatie    \\
        methoden er bestaan.                                             \\\hline
    \end{tabular}
  \end{center}
\end{table}

Tijdens de tweede iteratie is er gewerkt aan het opzetten van een
testopstelling met OpenRemote, de DreamPlug en de X10 hardware. Ook is er
onderzoek gedaan naar verschillende beveilingsmethodieken en oplossingen.

\begin{wrapfigure}{r}{0.4\textwidth}
  \begin{center}
    \includegraphics[width=0.20\textwidth]{android.pdf}
  \end{center}
  \caption{Android}
\end{wrapfigure}

Met de test opstelling is het mogelijk om een lamp
aan te sturen met de OpenRemote controller, welke draait op de
DreamPlug. Met de OpenRemote app is het mogelijk om met behulp van deze
opstelling een lamp aan te zetten via een X10 stopcontactmodule.

Het onderzoek over beveiliging concludeert dat de beste methode om
OpenRemote te beveiligen het gebruiken van SSL certificaten is. Deze
certificaten kunnen door zowel de server als de client gebruikt worden om
zich te authenticeren.

\subsection{Iteratie 3}
De derde iteratie heeft in het teken gestaan van de Proof-of-Concept die laat zien
dat het mogelijk is om met Android te verbinden naar een TomCat server en
te authenticeren door middel van SSL Client Certificaten.

\begin{table}[htpb]
  \caption{Eisen iteratie 3}
  \begin{center}
    \begin{tabular}{|| l ||}\hline
        Eis                                                              \\\hline\hline
        Als beheerder wil ik zien welke apparaten er zijn aangemeld      \\\hline
        Als systeem wil ik onderscheid zien tussen verschillende devices \\\hline
        Als gebruiker wil ik zien dat mijn telefoon uniek                \\ 
        identificeerbaar is                                              \\\hline
    \end{tabular}                                                         
  \end{center}                                                            
\end{table}                                                               

Voor de Proof-of-Concept zijn een aantal eisen. Als eerste moet de client
een aanvraag kunnen doen bij de server voor een certificaat. Deze aanvraag
moet in een lijstje te zien zijn waarna er een certificaat terug wordt
gestuurd naar de client. Dit certificaat wordt gegenereerd op de server en
met Base64 geencodeerd en overgestuurd naar de client.

De beheerder kan op de server een lijst met aanvragen zien evenals een
lijst met gebruikers die daadwerkelijk een certificaat hebben. Er is in deze
sprint nog niet gewerkt aan het dynamisch genereren van certificaten voor
gebruikers. Dit wordt in deze versie nog automatisch gedaan, in de veolgende
iteraties zal deze functionaliteit toegevoegd worden.

Er is ook voor gezorgd dat de gebruikers uniek identificeerbaar zijn. De server
en client moet kunnen zien dat de client uniek identificeerbaar is. Er wordt
geburik gemaakt van het 'serial' attribuut wat in elk certificaat aanwezig is.
Dit attribuut is een oplopende integer.

\subsection{Iteratie 4}
In de daaropvolgende iteratie is er gewerkt aan het implementeren van de
Proof-of-Concept in het OpenRemote project.

Er zijn een aantal onderdelen verbeterd om ze te kunnen implementeren in het
OpenRemote project. De beheerders
interface is een onderdeel hiervan. Deze interface is verbeterd, hij is
ontworpen (zie figuur ~\ref{fig:adminv1}) en er zijn acties aan gekoppeld 
waarmee de client goedgekeurd kan worden.

\begin{table}[htpb]
  \caption{Eisen iteratie 4}
  \begin{center}
    \begin{tabular}{|| l ||}\hline
        Eis                                                              \\\hline\hline
        Als beheerder wil ik sommige gebruikers toestemming geven in een \\ 
        webinterface                                                     \\\hline
        Als gebruiker wil ik met de android app van OpenRemote client    \\ 
        certificaten gebruiken om te verbinden met de server             \\\hline
        Als gebruiker wil ik met de android app van OpenRemote een client\\
        certificaat op kunnen halen van de server                        \\\hline
        Als beheerder wil ik een lijst zien in met gebruikers in een     \\ 
        webpagina op de openremote controller                            \\\hline
        Als beheerder wil ik clients toestemming kunnen geven door middel\\ 
        van een PIN uit te wisselen en te controleren in de web interface\\\hline
    \end{tabular}
  \end{center}
\end{table}

Van deze clients kan informatie getoond worden, namelijk het email adres,
de naam van de telefoon en een PIN. Deze PIN wordt gegenereerd door een
hash van de publieke sleutel te maken en daarvan de laatste 4 karakters te
laten zien.

\begin{figure}[h!]
  \centering
    \includegraphics[height=150pt,keepaspectratio]{adminv1.pdf}
  \caption{Ontwerp beheerderspanel}
  \label{fig:adminv1}
\end{figure}

De app stuurt sinds deze iteratie een CSR naar de server. De server
gebruikt deze CSR om een certificaat te maken. Op deze manier wordt de private
sleutel van de client nooit overgestuurd en kan deze ook niet onderschept
worden door kwaadwillenden.

\subsection{Iteratie 5}
\begin{table}[htpb]
  \caption{Eisen iteratie 5}
  \begin{center}
    \begin{tabular}{|| l ||}\hline
        Eis                                                              \\\hline\hline
        Als beheerder wil ik in kunnen loggen met de gegevens van        \\
        http://composer.openremote.org                                   \\\hline
        Als beheerder wil ik de toegang van clients kunnen intrekken van \\ 
        het systeem om de toegang te ontzeggen                           \\\hline
        Als beheerder wil ik de OpenRemote Controller op alle            \\ 
        verschillende platformen kunnen installeren                      \\\hline
        Als gebruiker wil ik de interface zo simpel mogelijk maken       \\\hline
    \end{tabular}
  \end{center}
\end{table}

In de vijfde iteratie is er voor gezorgd dat alle platformen ondersteund worden.
OpenRemote is een cross platform applicatie. De Controller draait op TomCat
welke in Java geschreven is. Dit betekent dat het draait op elk platform waar
Java draait. Echter hebben de wijzigingen, die in de vorige iteratie zijn
doorgevoert, er voor gezorgd dat het noodzakelijk is toegang te hebben tot
het 'openssl' commando. In deze iteratie is gewerkt aan het vervangen van het
'openssl' commando door de Bouncy Castle library, wat ervoor zorgt dat de
cryptografische acties gebeuren door middel van Java, in plaats van een aanroep
naar openssl. 

In figuur ~\ref{fig:tls} is te zien wat er gebeurd als er een
aanvraag binnenkomt bij de Controller. De client doet in het begin een aanvraag
voor toegang. Hiermee stuurt hij een CSR bestand op waar al zijn gegevens in staan,
welke de Controller gebruikt om een certificaat te maken. 

Als er een aanvraag binnen komt voegt de Controller de nieuwe client toe aan de
database. De gegevens in de database worden uit de CSR gehaald. vervolgens is de
client te zien in de beheerdersinterface. Hij is nog niet
goedgekeurd en heeft dus ook geen toegang tot het systeem.

Wanneer de beheerder de client goedgekeurd heeft, en desgewenst de pin ingevoerd
heeft, wordt het een certificaat gegenereert. Als eerste
wordt de private key van de CA opgehaald. Elk certificaat dat gemaakt wordt,
wordt ondertekend door de CA. Deze CA wordt ook geimporteerd in de OpenRemote
Controller en op die manier weet OpenRemote dat hij de clients kan vertrouwen.

Daarna wordt de private key samen met de CSR gebruikt om een certificaat te
cree"eren. Tevens moet er een serienummer meegegeven. Dit serienummer is een
uniek oplopend nummer waarmee de certificaten te identificeren zijn. Het
certificaat wat gemaakt is wordt deze in een keystore
opgeslagen, klaar om opgehaald te worden door de mobiele applicatie. Tevens wordt de
status van de client in de database aangepast.

Nu kan de gebruiker met de applicatie een certificaat ophalen en met dit
certificaat een veilige verbinding opzetten. Tevens kan TomCat het certificaat
gebruiken om de gebruiker te identificeren. In de database staat de 'dname'
welke ook in het certificaat staat. Op het moment dat een client een verbinding
maakt met TomCat, controleert TomCat de dname en kijkt in de database of deze
gebruiker toegang heeft tot het systeem. Mocht dat niet het geval zijn, zal hij
de gebruiker geen toegang tot het systeem geven.
Zie bijlage 2.

Tevens is er een
configuratiescherm gemaakt. In dit scherm kunnen een aantal dingen worden
ingesteld zoals onder andere het pad waar de CA opgeslagen wordt en of de PIN
verplicht gecontroleerd moet worden.

\begin{figure}[h!]
  \centering
    \includegraphics[width=0.7\textwidth,keepaspectratio]{adminv2config.pdf}
  \label{fig:adminv2config}
  \caption{Het configuratiescherm}
\end{figure}

Dit configuratiescherm en het user management scherm zijn beide beveiligd met
een gebruikersnaam en wachtwoordcombinatie. Deze combinatie is dezelfde als
waarmee ingelogd wordt op \url{http://composer.openremote.org/demo/}. 
Deze worden gecontrolleerd online als er ingelogd wordt. Mocht er geen
internetverbinding zijn wordt de cache uit de database gebruikt.

\subsection{Iteratie 6}
\begin{table}[htpb]
  \caption{Eisen iteratie 6}
  \begin{center}
    \begin{tabular}{|| l ||}\hline
        Eis                                                              \\\hline\hline
        Als beheerder wil ik authenticatie en authorisatie uit kunnen    \\
        schakelen om terug te vallen op de standaard onbeveiligde        \\ 
        verbinding                                                       \\\hline
        Als gebruiker wil ik alleen via SSL port \& client certificaten  \\ 
        bij de OpenRemote Controller kunnen komen                        \\\hline
        Als ontwikkelaar wil ik documentatie hebben over de implementatie\\ 
        van het authenticatie systeem                                    \\\hline
        Als ontwikkelaar wil ik een patch hebben van de toevoegingen ten \\
        opzichte van een OpenRemote release                              \\\hline
    \end{tabular}
  \end{center}
\end{table}

In de zesde iteratie van dit project is er gewerkt aan het opleveren van een
stabiele release voor de ontwikkelaars van OpenRemote. Zoals eerder vermeld is
er in dit project gewerkt aan het uitbreiden van een bestaand open source
project namelijk OpenRemote. 

Er is voor gezorgd dat het verbinden met SSL client certificaten verplicht
gesteld kan worden. Dit betekent dat vanaf deze iteratie het alleen mogelijk is
om te verbinden met de Controller met behulp van een bekend SSL Client
Certificaat. Er wordt,
als er verbinding is, gecontroleert of het client certificaat bekend
is bij de Controller. Zo ja, dan is er toegang, anders niet.

Ook is ervoor gezorgd dat de uitbreidingen op OpenRemote uit te zetten
zijn. Er is een checkbox bijgekomen in het configuratie paneel welke, als
deze uitgevinkt wordt, er voor zorgt dat TomCat niet meer controleert op client
certificaten. 

Tevens is er ook contact opgenomen met OpenRemote om de vorderingen te tonen.
Dit is gedaan via een bericht op het forum van OpenRemote welke te vinden is via
de volgende link 
\url{http://openremote.org/pages/viewpage.action?pageId=19439381&focusedCommentId=19440285#comment-19440285}. 
In de forum post is kort uitgelegd wat de veranderingen zijn waar aan gewerkt is
en hoe dit getest is. 
 
\subsection{Iteratie 7}

In deze iteratie is er gewerkt aan het implementeren van groepen in OpenRemote.
Deze functies houdt in dat men een groep toe kan wijzen aan een knop in de layout
designer, en een telefoon/tablet aan een groep toe kan wijzen. Op het moment dat er
een device en een knop in dezelfde groep zitten, zal de actie die aan de knop
hangt doorgevoerd worden.

\begin{table}[htpb]
  \caption{Eisen iteratie 7}
  \begin{center}
    \begin{tabular}{|| l ||}\hline
        Eis                                                              \\\hline\hline
        Als beheerder kan ik groepen aanmaken om gebruikers in te delen  \\\hline
        Als beheerder kan ik gebruikers toewijzen aan groepen om         \\ 
        gebruikers rechten te geven                                      \\\hline
        Als beheerder kan ik knoppen toewijzen aan groepen om groepen    \\ 
        rechten te geven                                                 \\\hline
    \end{tabular}
  \end{center}
\end{table}

De modeler is uitgebreid met de mogelijkheid om groepen aan te maken, en
vervolgens deze groepen toe te kunnen wijzen aan "besturingselementen". Er kan
nu een groep toegewezen worden aan een knop, een switch en een slider. In de XML
die vervolgens gegenereerd wordt zijn ook de groepen toegevoegd. 

In de controller kan een gebruiker toegewezen worden aan een groep. Deze groepen
zijn hetzelfde als de groepen die worden aangemaakt in modeler en worden
meegestuurd in de XML. Standaard heeft een gebruiker geen groep, en dus geen
rechten, en in het administrator paneel kan een groep toegewezen worden.

Mocht er toch op een knop gedrukt worden op de android applicatie wordt er een
melding getoond met daarin het bericht dat de gebruiker niet in de correcte
groep zit. 

\subsection{Iteratie 8}

\begin{table}[htpb]
  \caption{Eisen iteratie 8}
  \begin{center}
    \begin{tabular}{|| l ||}\hline
        Eis                                                              \\\hline\hline
        Als beheerder kan ik knoppen aan meerdere groepen toewijzen om zo\\
        meer gebruikers er rechten toe te geven                          \\\hline
        Als gebruiker wil ik de knoppen die ik niet aan kan sturen niet  \\ 
        kunnen bedienen                                                  \\\hline
    \end{tabular}
  \end{center}
\end{table}

In deze sprint was het belangrijk om de groepen te breiden. In de modeler was
het nog niet mogelijk om meerdere groepen te selecteren voor \'e\'en knop.
Daarom is er in deze iteratie een nieuwe scherm ontworpen en ge\"implementeerd
om gebruiksvriendelijk meerdere groepen te kunnen selecteren.

\begin{figure}[h!]
  \centering
    \includegraphics[width=0.6\textwidth,keepaspectratio]{groupselect.pdf}
  \label{fig:groupselect}
  \caption{Het groepenselecteerscherm}
\end{figure}

Ook is er in deze iteratie voor gezorgd dat de knoppen die niet aangestuurd
mogen worden, op Android ook daadwerkelijk gedisabled worden. Dit betekent dat
de knoppen donkergrijs worden gekleurd en niet meer in te drukken zijn.

%%
%Screenshot android applicatie
%%

\newpage
\section{Conclusies \& Aanbevelingen}
Samenvattend is het DomoTop project succesvol afgerond. OpenRemote is nu beveiligd
via SSL/TLS en client certificaten. Dit maakt het mogelijk om apparaten te
authoriseren en te kiezen of deze apparaten toegang krijgen tot de
OpenRemote Controller. Tevens is aan het eind van het DomoTop project ook de
mogelijkheid om groepen te gebruiken toegevoegd. Zo kunnen er groepen worden aangemaakt en
gekoppeld worden aan knoppen, sliders en switches. Daarna kan men, in de OpenRemote
Controller backend, de apparaten koppelen aan een bepaalde groep. Op deze
manier wordt er ook gebruik gemaakt van authorisatie.

Kortom het beveiligen van OpenRemote is gelukt en de gemaakte code kan weer
opgelevert worden aan de OpenRemote community, waarna ook de rest van de wereld
gebruik kan maken van de verbetering in beveiling die ontwikkelt zijn in dit
project. 

Binnen het DomoTop project zijn er punten die nog verbeterd kunnen worden. Het eerste waar
aan gewerkt zou kunnen worden is het implementeren van de beveilings
verbeteringen in het OpenRemote project in de iOS client. Op dit moment zijn de
authenticatie en authorisatie verbeteringen nog niet ge\"implementeerd in de iOS
client.

Een goede uitbreiding aan dit project zou een Windows Phone 7 (WP7) client zijn.
OpenRemote heeft op dit moment alleen een Android, web en iOS client. 

Op het moment dat er uitbreidingen of verbeteringen uitgevoerd gaan worden op
dit project is het verstandig dat er kennis is van SSL/TLS. Er moet kennis zijn
van het cree"eren van CSR's en het ondertekenen daarvan. Tevens is het
verstandig kennis op te doen van Java en in het bijzonder Java op Android en in
TomCat.

\newpage
\section{Verklarende woordenlijst}

\begin{longtable}{|| l | l ||}\hline
    Begrip           & Omschrijving                                         \\\hline\hline
    CAN-Bus          & CAN-bus is een standaard voor in voertuigen          \\
                     & om microcontrollers en apparaten met elkaar te       \\
                     & laten communiceren zonder een host-computer.        \\\hline
    OpenRemote       & Een domotica pakket waarmee je meerdere domotica     \\
                     & protocollen aan kan sturen.                          \\\hline
    Android          & Een besturingssysteem voor mobiele telefoons en      \\
                     & tablets gemaakt door Google.                         \\\hline
    iOS              & Een besturingssysteem voor mobiele telefoons en      \\
                     & tablets gemaakt door Apple.                          \\\hline
    GitHub           & Een online git hosting platform.                     \\\hline
    SCRUM            & Een ontwikkelmethodiek die werkt met meerdere        \\
                     & iteraties om zo flexibel te kunnen                   \\
                     & ontwikkelen.                                         \\\hline
    PlugTop          & Een energiezuinige computer in het formaat van een   \\
                     & forse adapter.                                       \\\hline
    Multithreaded    & Meer dan \'e\'en thread, dit kan de efficientie van de   \\
                     & hardware ten goede komen.                            \\\hline
    CA               & CA staat voor Certificate authority en is bedoeld om \\
                     & certificaat aanvragen goed te keuren en een          \\
                     & certificaten te beheren en te generen.               \\\hline
    TomCat Web       & TomCat heeft een Web Application Manager waarbij je  \\
    Application      & eenvoudig via een web interface je applicaties (WAR  \\
    Manager          & bestanden) te beheren, updaten, uploaden.            \\\hline
    Java Servlet     & Een servlet is een in Java geschreven programma dat  \\
                     & op een server draait. De servlet maakt hierbij       \\
                     & gebruik van een aantal diensten die de               \\
                     & webcontainer biedt, zoals het afhandelen van de      \\
                     & communicatie met de client.                          \\\hline
    Model View       & Model-view-controller (of MVC) is een ontwerppatroon \\
    Controller       & dat het ontwerp van complexe toepassingen opdeelt in \\
    software         & drie eenheden met verschillende                      \\
    Architectuur     & verantwoordelijkheden: datamodel, datapresentatie en \\
                     & applicatielogica. Het scheiden van deze              \\
                     & verantwoordelijkheden bevordert de leesbaarheid en   \\
                     & herbruikbaarheid van code.                           \\\hline
    KeyStore         & Een Java keystore (JKS) is een bestand waar          \\
                     & beveiligingscertificaten, aanvraag certificaten of   \\
                     & publieke sleutels worden opgeslagen                  \\
                     & wat gebruikt wordt met SSL encryptie.                \\\hline
    TrustStore       & Zie: KeyStore, een TrustStore bevat certificaten     \\
                     & die zijn geaccepteerd om verbinding mee op te zetten.\\
                     & Of een CA ceritificaat waarbij de ondertekende       \\
                     & certificaten vertrouwd  zijn.                        \\\hline
    Self-signed      & Een certificaat kan self-signed zijn,                \\
                     & wat erop duidt dat het certificaat door zichtzelf    \\
                     & ondertekend is.                                      \\\hline
    Proof-of-Concept & Een applicatie die laat zien dat een bepaalde        \\
                     & techniek gebruikt kan worden. Deze applicatie laat   \\
                     & zien dat de techniek werkt.                   \\\hline
    CSR              & Certification Sign Request, met behulp van dit       \\
                     & bestand kan de CA een certificaat genereren.         \\\hline
\end{longtable}

\newpage
\section{Bronnen}
[Wordt nog bijgewerkt]

\newpage
\section{Bijlagen}

In dit hoofdstuk komen de bijlagen te staan. Elke bijlage is genummerd en
dit nummer kan worden in het document als referentie.

\subsection{Bijlage 1: Plan van Aanpak}

\newpage

\begin{landscape}
\subsection{Bijlage 2: }
\begin{figure}[h!]
  \centering
    \includegraphics[height=0.80\textheight,keepaspectratio]{DomoTopSequenceDiagram_without_title.pdf}
  \label{fig:tls}
  \caption{DomoTop Sequence Diagram - Afhandelen van een request via een TLS
  capable OpenRemote Controller}
\end{figure}
\end{landscape}


\newpage
\section{Verantwoording individuele bijdrage}

\begin{tabular}{|| l | c | c ||}\hline
    Onderdeel              &   \multicolumn{2}{|c||}{Verantwoordelijke} \\\hline
                           & Melroy van den Berg & Vincent Kriek        \\\hline\hline
    Voorwoord              &                     &  X                   \\\hline
    Samenvatting           &       X             &                      \\\hline
    Inleiding              &       X             &                      \\\hline
    De plaats van de...    &                     &  X                   \\\hline
    Probleemanalyse        &                     &  X                   \\\hline
    Plan van Aanpak        &       X             &                      \\\hline
    Methoden en Technieken &       X             &  X                   \\\hline
    Uitvoering             &       X             &                      \\\hline
    Resultaten             &                     &  X                   \\\hline
    Conclusies             &       X             &                      \\\hline
    Aanbeveling            &                     &  X                   \\\hline
\end{tabular}              


\end{document}
